% Created 2020-12-03 Thu 09:05
% Intended LaTeX compiler: pdflatex
\documentclass[letterpaper]{tufte-handout}
\usepackage{graphicx}
\usepackage{grffile}
\usepackage{longtable}
\usepackage{wrapfig}
\usepackage{rotating}
\usepackage[normalem]{ulem}
\usepackage{amsmath}
\usepackage{textcomp}
\usepackage{amssymb}
\usepackage{capt-of}
\usepackage{hyperref}
\usepackage{minted}
\usepackage{fontspec}
\defaultfontfeatures{Ligatures=TeX}
\usepackage[small,sf,bf]{titlesec}
\ifx\ifxetex\ifluatex\else % if lua- or xelatex http://tex.stackexchange.com/a/140164/1913
\newcommand{\textls}[2][5]{%
\begingroup\addfontfeatures{LetterSpace=#1}#2\endgroup
}
\renewcommand{\allcapsspacing}[1]{\textls[15]{#1}}
\renewcommand{\smallcapsspacing}[1]{\textls[10]{#1}}
\renewcommand{\allcaps}[1]{\textls[15]{\MakeTextUppercase{#1}}}
\renewcommand{\smallcaps}[1]{\smallcapsspacing{\scshape\MakeTextLowercase{#1}}}
\renewcommand{\textsc}[1]{\smallcapsspacing{\textsmallcaps{#1}}}
\fi
\setmainfont{Adobe Caslon Pro}
\setmonofont{Anonymous Pro}
\usepackage{booktabs} % book-quality tables
\usepackage{units}    % non-stacked fractions and better unit spacing
\usepackage{multicol} % multiple column layout facilities
\usepackage{lipsum}   % filler text
\usepackage{fancyvrb} % extended verbatim environments
\fvset{fontsize=\normalsize}% default font size for fancy-verbatim environments
\newcommand{\doccmd}[1]{\texttt{\textbackslash#1}}% command name -- adds backslash automatically
\newcommand{\docopt}[1]{\ensuremath{\langle}\textrm{\textit{#1}}\ensuremath{\rangle}}% optional command argument
\newcommand{\docarg}[1]{\textrm{\textit{#1}}}% (required) command argument
\newcommand{\docenv}[1]{\textsf{#1}}% environment name
\newcommand{\docpkg}[1]{\texttt{#1}}% package name
\newcommand{\doccls}[1]{\texttt{#1}}% document class name
\newcommand{\docclsopt}[1]{\texttt{#1}}% document class option name
\newenvironment{docspec}{\begin{quote}\noindent}{\end{quote}}% command specification environment
\usepackage{babel}
\usepackage[fixlanguage]{babelbib}
\selectbiblanguage{english}
\usepackage[sort,round]{natbib}
\usepackage[nottoc]{tocbibind}
\usepackage{csquotes}
\usepackage{multirow}
\hypersetup{colorlinks}
\definecolor{sectionColor}{rgb}{0.663,0,0.064} %% red
\definecolor{citeColor}{rgb}{0.753,0.18,0.114} %% orange
\subsectionfont{\color{sectionColor}}
\sectionfont{\color{sectionColor}}
\hypersetup{%
pdfborder = {0 0 0},
bookmarksdepth = section,
citecolor = sectionColor,
linkcolor = sectionColor,
urlcolor = citeColor,
}
\newcommand{\degC}{$^\circ$C{}}
\usepackage{fancyhdr} \pagestyle{fancyplain} \fancyhf{} \renewcommand{\headrulewidth}{0pt} \lhead{\scriptsize{SERGIO-FELICIANO MENDOZA-BARRERA}} \rhead{\scriptsize{PERSONAL RESEARCH $\cdot\ 2020\ \cdot$ GLOBAL LABS $\cdot$ MEXICO}}\fancyfoot[RO, LE]{\thepage}
\newcommand\at[2]{\left.#1\right|_{#2}}
\newcommand\HHI{\mathit{HHI}}
\newcommand\CR{\mathit{CR}}
\definecolor{bg}{rgb}{0.973, 0.973, 0.973}
\author{Sergio-Feliciano Mendoza-Barrera}
\date{\today}
\title{100 days of code with Julia}
\hypersetup{
 pdfauthor={Sergio-Feliciano Mendoza-Barrera},
 pdftitle={100 days of code with Julia},
 pdfkeywords={R, data science, research, methodology, julia},
 pdfsubject={Julia 100 days training},
 pdfcreator={Emacs 27.1 (Org mode 9.4)},
 pdflang={English}}
\begin{document}

\maketitle
\begin{abstract}
The today's topic is metaprogramming basics.
\end{abstract}

\section{Metaprogramming in Julia. Day 1}
\label{sec:org05ee56a}

Julia code can be represented as a data structure of the language
itself. This allows a program to transform and generate its own code
\citep[p. 204]{lauwens2020}.

\subsection{Expressions}
\label{sec:orgf3273b3}

Every Julia program starts as a string

\begin{minted}[,framesep=2mm, baselinestretch=1.2, linenos, fontsize=\footnotesize, breaklines=true, bgcolor=bg, style=xcode]{julia}
prog = "1 + 2"
ex = Meta.parse(prog)
\end{minted}

we can get the type of the variable with \texttt{typeof} as usual and \texttt{dump}
the tree structure\footnote{The \texttt{dump} function displays \texttt{expr} objects
with annotations.}.

\begin{minted}[,framesep=2mm, baselinestretch=1.2, linenos, fontsize=\footnotesize, breaklines=true, bgcolor=bg, style=xcode]{julia}
typeof(ex)
dump(ex)
\end{minted}

\begin{verbatim}
Expr
Expr
  head: Symbol call
  args: Array{Any}((3,))
    1: Symbol +
    2: Int64 1
    3: Int64 2
\end{verbatim}


\noindent Expressions can be constructed directly by prefixing with \texttt{:}
inside parentheses or using a \texttt{quote} block

\begin{minted}[,framesep=2mm, baselinestretch=1.2, linenos, fontsize=\footnotesize, breaklines=true, bgcolor=bg, style=xcode]{julia}
ex = quote
    1+2
end
\end{minted}

Now, \emph{Julia can evaluate an expression object} using \texttt{eval}

\begin{minted}[,framesep=2mm, baselinestretch=1.2, linenos, fontsize=\footnotesize, breaklines=true, bgcolor=bg, style=xcode]{julia}
eval(ex)
\end{minted}

\begin{verbatim}
3
\end{verbatim}


Every module has its own \texttt{eval} function that evaluates expressions in
its scope\footnote{\emph{WARNING}: When you are using a lot of calls to the
function \texttt{eval}, often this means that something is wrong. \texttt{eval} is
considered \emph{evil}.}.

\section{Macros. Day 2}
\label{sec:org67109ea}

\newthought{Macros can include generated code in a program}. A macro
maps a tuple of \texttt{Expr} objects directly to a compiled expression:

Here is a simple macro

\begin{minted}[,framesep=2mm, baselinestretch=1.2, linenos, fontsize=\footnotesize, breaklines=true, bgcolor=bg, style=xcode]{julia}
macro containervariable(container, element)
    return esc(:($(Symbol(container,element)) = $container[$element]))
end
\end{minted}

Macros are called by prefixing their name with the \texttt{@} (at-sign). The
macro call \texttt{@containervariable letters 1} is replaced by:

\begin{verbatim}
:(letters1 = letters[1])
\end{verbatim}

\texttt{@macroexpand @containervariable letters 1} returns this expression
which is extremely useful for debugging.

\begin{minted}[,framesep=2mm, baselinestretch=1.2, linenos, fontsize=\footnotesize, breaklines=true, bgcolor=bg, style=xcode]{julia}
@macroexpand @containervariable letters 1
\end{minted}

This example illustrates how a macro can access the name of its
arguments, something a function can’t do. The return expression needs
to be \emph{escaped} with \texttt{esc} because it has to be resolved in the macro
call environment\footnote{\emph{Note:} Why macros? Macros generate and include
fragments of customized code during parse time, thus before the full
program is run.}.

\subsection{A concrete example}
\label{sec:org63b5ab7}

\newthought{A concrete example of the macro usage} is show here
\cite{MIT18S191}, in this case we are going to use the \texttt{@elapsed} macro
over the \texttt{peakflops} function

\begin{minted}[,framesep=2mm, baselinestretch=1.2, linenos, fontsize=\footnotesize, breaklines=true, bgcolor=bg, style=xcode]{julia}
peakflops()
\end{minted}

\begin{verbatim}
5.892558158665142e10
\end{verbatim}


now we can see how long the operation took

\begin{minted}[,framesep=2mm, baselinestretch=1.2, linenos, fontsize=\footnotesize, breaklines=true, bgcolor=bg, style=xcode]{julia}
@elapsed peakflops()
\end{minted}

\begin{verbatim}
0.346514384
\end{verbatim}


Internally, Julia take this piece of code and transform in another
piece of code

\begin{minted}[,framesep=2mm, baselinestretch=1.2, linenos, fontsize=\footnotesize, breaklines=true, bgcolor=bg, style=xcode]{julia}
@macroexpand @elapsed peakflops()
\end{minted}

\begin{verbatim}
quote
    #= timing.jl:231 =#
    while false
        #= timing.jl:231 =#
    end
    #= timing.jl:232 =#
    local var"#5#t0" = Base.time_ns()
    #= timing.jl:233 =#
    peakflops()
    #= timing.jl:234 =#
    (Base.time_ns() - var"#5#t0") / 1.0e9
end
\end{verbatim}

the comments indicate the place where Julia inserts extra code for
this specific macro

\begin{minted}[,framesep=2mm, baselinestretch=1.2, linenos, fontsize=\footnotesize, breaklines=true, bgcolor=bg, style=xcode]{julia}
Base.remove_linenums!(@macroexpand @elapsed peakflops())
\end{minted}

\begin{verbatim}
quote
    while false
    end
    local var"#7#t0" = Base.time_ns()
    peakflops()
    (Base.time_ns() - var"#7#t0") / 1.0e9
end
\end{verbatim}


\bibliographystyle{newapa}
\bibliography{code}
\end{document}
